\chapter{Global Settings}
\label{global}

\section{Polyphonic / Legato}
\label{legato}

Odin 2 is capable of two play modes: \fat{polyphonic} or \fat{legato}.

\vspace{3mm}
In the default polyphonic mode, you can play up to 24 voices. When you reached the voice limit, the oldest note that is not still held down will be stolen and used as the new voice.

\vspace{3mm}
In legato mode, you have only one voice to play with. This can be very useful if you don't want notes to overlap, for example in a lead or bass type of sound. In legato mode, the \hyperref[ADSR]{envelopes} perform a "soft reset". Instead of starting the attack from zero, the last value output by the envelope will be used as the start of the attack section. This way, clicky noises can be avoided for example in the Amp Envelope.

\vspace{3mm}
The switch for the play modes can be found in the top-right corner of the GUI.


\audioparameter{Poly / Legato}{0}{1}{
    \begin{center}
        \includegraphics[width=0.15\textwidth]{graphics/poly_legato.png}
    \end{center}
    Toggles between polyphonic and legato play modes.
}

\clearpage
\section{Unison}
\label{unison}

\begin{center}
    \includegraphics[width=0.4\textwidth]{graphics/unison.png}
\end{center}

The Unison feature makes it possible to trigger a stack of voices together instead of a single voice. These voices can be slightly detuned and spread over the stereo field, to generate a much wider, bigger sound. While detune and width parameters are available as dedicated controls, you can modulate any parameter for each unison voice independently with the modulation destination "Unison Index" (see Section \ref{mod_destinations}).

\vspace{3mm}
The controls for Unison can be found in the top-left corner of the GUI.

\vspace{3mm}
\begin{tcolorbox}[colback=yellow!10!white,
        colframe=white!20!black,
        center,
        valign=top,
        halign=left,
        center title,
        width=\textwidth]

    Since the implementation of Unison in Odin 2 literally triggers multiple voices, using high Unison counts uses lots of CPU time. Use it with care, if you run into performance issues, you can always bounce the track to audio in your DAW.
\end{tcolorbox}

\audioparameter{Unison Count}{0}{0}{
    Determines how many voices are triggered when you press a button. 

    Try to keep the Unison Count low if you're experiencing performance issues.
}

\audioparameter{Unison Detune}{0}{1}{
    Controls the amount of detune for the voices in the unison stack.
}

\audioparameter{Unison Width}{0}{1}{
    Controls the stereo spread of the voices in the unison stack. Having this value at zero will put all voices in the center of the stereo field. Having the value at one puts the first voices hard-left and the last on hard-right. The voices inbetween are spread linearly inbetween
}

\section{Glide and Master}
\label{glide}

The last two remaining parameters, Glide and Master are located in the bottom left corner of the GUI.

\begin{center}
    \includegraphics[height=0.2\textwidth]{graphics/master_glide.png}
\end{center}

\audioparameter{Glide}{1}{1}{
    Makes the pitch of newly triggered voices glide from the frequency of the last note to the frequency of the current note. The glide curve is exponential.
}

\audioparameter{Master}{1}{1}{
    Controls the output gain of the synthesizer.
}
